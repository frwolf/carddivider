% Basic settings for this card set
\renewcommand{\cardcolor}{seaside}
\renewcommand{\cardextension}{Erweiterung I}
\renewcommand{\cardextensiontitle}{Seaside}
\renewcommand{\seticon}{seaside.png}

\clearpage
\newpage
\section{\cardextension \ - \cardextensiontitle \ 2. Edition (Rio Grande Games 2022)}

\begin{tikzpicture}
	\card
	\cardstrip
	\cardbanner{banner/white.png}
	\cardicon{icons/coin.png}
	\cardprice{2}
	\cardtitle{\tiny{Eingeborenendorf}}
	\cardcontent{Wenn du dein erstes \emph{EINGEBORENENDORF} nimmst oder kaufst, erhältst du ein Eingeborenen-Tableau und legst es vor dir ab. 

	\medskip

	Immer wenn du ein \emph{EINGEBORENENDORF} ausspielst, wählst du genau eine der beiden Anweisungen und führst sie wenn möglich aus. Du darfst eine Anweisung auch wählen, wenn du sie nicht ausführen kannst. Karten, die du auf das Tableau legst, werden immer verdeckt abgelegt. Du darfst dir jederzeit die Karten auf deinem Tableau ansehen. 

	\medskip

	Die ausgespielte Aktionskarte \emph{EINGEBORENENDORF} legst du in der Aufräumphase ab. Alle Karten auf dem Tableau gehören auch zum Kartensatz eines Spielers. Alle Karten auf den Tableaus werden bei Spielende mit berücksichtigt. }
\end{tikzpicture}
\hspace{-0.6cm}
\begin{tikzpicture}
	\card
	\cardstrip
	\cardbanner{banner/orange.png}
	\cardicon{icons/coin.png}
	\cardprice{2}
	\cardtitle{Hafen}
	\cardcontent{Lege eine Handkarte verdeckt unter den \emph{HAFEN}. Diese und der \emph{HAFEN} werden in der Aufräumphase nicht abgelegt. Nimm zu Beginn deines nächsten Zuges die zur Seite gelegte Karte auf die Hand. Lege den \emph{HAFEN} in der Aufräumphase ab.}
\end{tikzpicture}
\hspace{-0.6cm}
\begin{tikzpicture}
	\card
	\cardstrip
	\cardbanner{banner/orange.png}
	\cardicon{icons/coin.png}
	\cardprice{2}
	\cardtitle{Leuchtturm}
	\cardcontent{Zwischen deinem Spielen des \emph{LEUCHTTURMS} und dem Beginn deines nächsten Zuges bist du nicht betroffen, wenn Mitspieler Angriffskarten spielen (sogar wenn du das möchtest). Zusätzlich darfst du natürlich weiterhin auf von Mitspielern gespielte Angriffskarten mit dem Aufdecken von Reaktionskarten reagieren. Lege den \emph{LEUCHTTURM} in der Aufräumphase des nächsten Zuges ab.}
\end{tikzpicture}
\hspace{-0.6cm}
\begin{tikzpicture}
	\card
	\cardstrip
	\cardbanner{banner/orange.png}
	\cardicon{icons/coin.png}
	\cardprice{3}
	\cardtitle{Affe}
	\cardcontent{Dies gilt auch für Karten, die der Spieler in Zügen anderer Mitspieler nimmt, wie z.B. einen \emph{FLUCH} in deinem Zug durch die \emph{HEXE}.}
\end{tikzpicture}
\hspace{-0.6cm}
\begin{tikzpicture}
	\card
	\cardstrip
	\cardbanner{banner/white.png}
	\cardicon{icons/coin.png}
	\cardprice{3}
	\cardtitle{Ausguck}
	\cardcontent{Sieh dir erst alle 3 Karten an, bevor du die Anweisungen ausführst. Solltest du weniger als 3 Karten im Nachziehstapel haben, auch nachdem du ggf. den Ablagestapel gemischt hast, führst du die Anweisungen der Reihenfolge nach aus. Anweisungen, für die es keine Karten mehr im Stapel gibt, entfallen.}
\end{tikzpicture}
\hspace{-0.6cm}
\begin{tikzpicture}
	\card
	\cardstrip
	\cardbanner{banner/goldorange.png}
	\cardicon{icons/coin.png}
	\cardprice{3}
	\cardtitle{Astrolabium}
	\cardcontent{Durch das \emph{ASTROLABIUM} erhältst du +1 Kauf und +\coin[1] in dem Zug, indem du es spielst, und in deinem nächsten Zug ebenfalls +1 Kauf und +\coin[1].}
\end{tikzpicture}
\hspace{-0.6cm}
\begin{tikzpicture}
	\card
	\cardstrip
	\cardbanner{banner/orange.png}
	\cardicon{icons/coin.png}
	\cardprice{3}
	\cardtitle{Fischerdorf}
	\cardcontent{Die Zahl deiner Aktionen erhöht sich um 2 und du erhältst +\coin[1].

	\medskip

	In deinem nächsten Zug erhältst du eine zusätzliche Aktion und +\coin[1]. }
\end{tikzpicture}
\hspace{-0.6cm}
\begin{tikzpicture}
	\card
	\cardstrip
	\cardbanner{banner/white.png}
	\cardicon{icons/coin.png}
	\cardprice{3}
	\cardtitle{Lagerhaus}
	\cardcontent{Ziehe drei Karten und lege dann drei Handkarten ab. Dann darfst du eine weitere Aktionskarte spielen. Hast du nach dem Ziehen 3 oder weniger Karten auf der Hand, legst du alle Handkarten ab.}
\end{tikzpicture}
\hspace{-0.6cm}
\begin{tikzpicture}
	\card
	\cardstrip
	\cardbanner{banner/white.png}
	\cardicon{icons/coin.png}
	\cardprice{3}
	\cardtitle{Schmuggler}
	\cardcontent{Hat der rechts von dir sitzende Mitspieler in seinem letzten Zug eine Karte mit Kosten von \coin[6] oder weniger genommen, nimmst du dir eine gleiche Karte vom Vorrat. Hat der Spieler mehrere Karten genommen, darfst du wählen, welche du nimmst. Da der \emph{SCHMUGGLER} keine Angriffskarte ist, dürfen keine Reaktionskarten aufgedeckt werden.
	
	\medskip
	
	Karten aus anderen Erweiterungen, die Kostenbestandteile außer \coin haben, kosten niemals weniger als \coin[6]. }
\end{tikzpicture}
\hspace{-0.6cm}
\begin{tikzpicture}
	\card
	\cardstrip
	\cardbanner{banner/white.png}
	\cardicon{icons/coin.png}
	\cardprice{3}
	\cardtitle{Seekarte}
	\cardcontent{Wenn du eine gleiche Karte wie die aufgedeckte im Spiel hast, einschließlich der gespielten \emph{SEEKARTE} oder einer Dauerkarte, die du aus einem früheren Zug im Spiel hast, nimm die aufgedeckte Karte auf deine Hand. Ansonsten legst du die aufgedeckte Karte oben auf deinen Nachziehstapel zurück.}
\end{tikzpicture}
\hspace{-0.6cm}
\begin{tikzpicture}
	\card
	\cardstrip
	\cardbanner{banner/white.png}
	\cardicon{icons/coin.png}
	\cardprice{4}
	\cardtitle{\scriptsize{Beutelschneider}}
	\cardcontent{Alle Mitspieler müssen ein \emph{KUPFER} aus der Hand ablegen. Wer kein \emph{KUPFER} ablegen kann, deckt seine Handkarten auf. Die Mitspieler dürfen auf das Spielen des \emph{BEUTELSCHNEIDERS} mit entsprechenden Reaktionskarten reagieren.}
\end{tikzpicture}
\hspace{-0.6cm}
\begin{tikzpicture}
	\card
	\cardstrip
	\cardbanner{banner/orange.png}
	\cardicon{icons/coin.png}
	\cardprice{4}
	\cardtitle{Blockade}
	\cardcontent{Die genommene Karte kommt aus dem Vorrat und wird zur Seite gelegt. Lege sie auf die \emph{BLOCKADE}, damit du weißt, für welche Karte die \emph{BLOCKADE} gilt. Falls die genommene Karte aus irgendeinem Grund nicht zur Seite gelegt bleibt (z.B. weil du sie mit einem \emph{WACHTURM} aus \emph{Blütezeit} entsorgst), wird die Blockade im aktuellen Zug abgeräumt. Ist die Karte zur Seite gelegt, nimmst du sie in deinem nächsten Zug auf deine Hand. Bis dahin gilt: Nehmen Mitspieler eine gleiche Karte in ihren Zügen, nehmen sie zusätzlich einen \emph{FLUCH}.}
\end{tikzpicture}
\hspace{-0.6cm}
\begin{tikzpicture}
	\card
	\cardstrip
	\cardbanner{banner/orange.png}
	\cardicon{icons/coin.png}
	\cardprice{4}
	\cardtitle{\scriptsize{Gezeitenbecken}}
	\cardcontent{Wenn du diese Karte spielst, ziehst du 3 Karten und erhältst +1 Aktion, aber zu Beginn deines nächsten Zuges musst du 2 Karten ablegen. Wenn du nur 1 Karte auf deiner Hand hast, legst du diese eine Karte ab. Und wenn du keine Karten auf deiner Hand hast, legst du keine ab. Wenn du mehrere Dauerkarten im Spiel hast, deren Anweisungen zu Beginn deines nächsten Zuges wirken, kannst du sie beliebig sortieren. Hast du zum Beispiel vier \emph{GEZEITENBECKEN} und eine \emph{WERFT}, könntest du alle deine Karten mit den \emph{GEZEITENBECKEN} ablegen und dann Karten für die \emph{WERFT} ziehen.}
\end{tikzpicture}
\hspace{-0.6cm}
\begin{tikzpicture}
	\card
	\cardstrip
	\cardbanner{banner/whitegreen.png}
	\cardicon{icons/coin.png}
	\cardprice{4}
	\cardtitle{Insel}
	\cardcontent{Die \emph{INSEL} ist eine kombinierte Aktions- und Punktekarte. Sie kann in der Aktionsphase eingesetzt werden und ist zusätzlich 2 Punkte wert. Wenn du deine erste \emph{INSEL} nimmst oder kaufst, erhältst du ein Insel-Tableau und legst es vor dir ab. 

	\medskip

	Immer wenn du eine \emph{INSEL} spielst, legst du die ausgespielte \emph{INSEL} und eine beliebige Handkarte offen auf dein Insel-Tableau. Dort verbleiben sie bis zum Spielende. Wenn du mindestens eine Karte auf der Hand hast, \emph{musst} du eine Handkarte auf dein Insel-Tableau legen. Wenn du keine Karte auf der Hand hast, nachdem du die \emph{INSEL} ausgespielt hast, legst du nur die \emph{INSEL} auf das Tableau. 

	\medskip

	Bei Spielende nimmst du alle Karten vom Insel-Tableau zu deinen Karten. }
\end{tikzpicture}
\hspace{-0.6cm}
\begin{tikzpicture}
	\card
	\cardstrip
	\cardbanner{banner/orange.png}
	\cardicon{icons/coin.png}
	\cardprice{4}
	\cardtitle{Karawane}
	\cardcontent{Die \emph{KARAWANE} wird in der Aufräumphase nicht abgelegt. Ziehe zu Beginn des nächsten Zuges eine Karte und lege die \emph{KARAWANE} in der Aufräumphase jenes Zuges ab.}
\end{tikzpicture}
\hspace{-0.6cm}
\begin{tikzpicture}
	\card
	\cardstrip
	\cardbanner{banner/white.png}
	\cardicon{icons/coin.png}
	\cardprice{4}
	\cardtitle{\footnotesize{Müllverwerter}}
	\cardcontent{Du musst eine Karte entsorgen, falls du eine auf der Hand hast. Pro \coin[1], das die entsorgte Karte kostet, erhältst du +\coin[1]. Wenn du keine Karte entsorgen kannst, erhältst du kein zusätzliches Geld.}
\end{tikzpicture}
\hspace{-0.6cm}
\begin{tikzpicture}
	\card
	\cardstrip
	\cardbanner{banner/white.png}
	\cardicon{icons/coin.png}
	\cardprice{4}
	\cardtitle{Schatzkarte}
	\cardcontent{Nur wenn du zusätzlich zu der ausgespielten \emph{SCHATZKARTE} noch eine weitere auf der Hand hast und beide entsorgst, erhältst du 4 \emph{Gold}. Sollten weniger als  4 \emph{Gold} im Vorrat sein, nimmst du dir so viele Goldkarten wie vorhanden sind. Lege alle auf diese Weise erhaltenen Goldkarten verdeckt auf den Nachziehstapel. Solltest du nur eine \emph{SCHATZKARTE} auf der Hand haben und diese spielen, entsorgst du diese Karte, erhältst aber nichts dafür.}
\end{tikzpicture}
\hspace{-0.6cm}
\begin{tikzpicture}
	\card
	\cardstrip
	\cardbanner{banner/orange.png}
	\cardicon{icons/coin.png}
	\cardprice{4}
	\cardtitle{Seefahrerin}
	\cardcontent{Wenn du eine Dauerkarte in dem Zug nimmst, in dem du die \emph{SEEFAHRERIN} spielst, ist das Spielen der Dauerkarte optional. Diese Karte wirkt kumulativ: Wenn du zwei \emph{SEEFAHRERINNEN} spielst, darfst du bis zu zwei genommene Dauerkarten spielen. Allerdings kannst du nicht mit zwei \emph{SEEFAHRERINNEN} ein und dieselbe Dauerkarte zweimal nacheinander spielen.
	
	\medskip

	Die \emph{SEEFAHRERIN} betrifft alle von dir genommenen Dauerkarten, also z.B. gekaufte oder mit Karten wie der \emph{WERKSTATT} genommene. Nimmst du eine Dauerkarte in deiner Kaufphase, kannst du sie durch die \emph{SEEFAHRERIN} spielen, obwohl du in deiner Kaufphase bist. Gibt dir eine solche Karte +Aktionen, darfst du dadurch keine Aktionskarten in deiner Kaufphase spielen. Wenn du dadurch Geldkarten ziehst, darfst du sie nur spielen, wenn du noch keine Karten gekauft hast. Die Fähigkeit Dauerkarten zu spielen, hast du nur in dem Zug, in dem du die \emph{SEEFAHRERIN} spielst. In deinem nächsten Zug erhältst du einfach nur +\coin[2] und darfst eine Karte aus deiner Hand entsorgen. }
\end{tikzpicture}
\hspace{-0.6cm}
\begin{tikzpicture}
	\card
	\cardstrip
	\cardbanner{banner/orange.png}
	\cardicon{icons/coin.png}
	\cardprice{5}
	\cardtitle{\footnotesize{Aussenposten}}
	\cardcontent{Der \emph{AUSSENPOSTEN} kann nur genutzt werden, wenn der Spieler vor dem aktuellen Zug nicht selbst an der Reihe war (er also nicht gerade schon einen Extrazug ausführt). Der Spieler zieht am Ende der Aufräumphase des aktuellen Zugs nur 3 statt der üblichen 5 Karten und führt danach einen Extrazug aus. Abgesehen von der kleineren Anzahl Handkarten ist das ein normaler Zug. Extrazüge zählen bei Spielende nicht mit, wenn es zu einem Gleichstand kommt. Am Ende des Extrazuges wird der \emph{AUSSENPOSTEN} abgelegt und 5 Karten werden nachgezogen.}
\end{tikzpicture}
\hspace{-0.6cm}
\begin{tikzpicture}
	\card
	\cardstrip
	\cardbanner{banner/white.png}
	\cardicon{icons/coin.png}
	\cardprice{5}
	\cardtitle{Bazar}
	\cardcontent{Du \emph{musst} eine Karte nachziehen, hast 2 zusätzliche Aktionen und erhältst +\coin[1].}
\end{tikzpicture}
\hspace{-0.6cm}
\begin{tikzpicture}
	\card
	\cardstrip
	\cardbanner{banner/orange.png}
	\cardicon{icons/coin.png}
	\cardprice{5}
	\cardtitle{\footnotesize{Handelsschiff}}
	\cardcontent{Du erhältst +\coin[2]. Zu Beginn deines nächsten Zuges erhältst du +\coin[2]. Lege das \emph{HANDELSSCHIFF} in der Aufräumphase jenes Zuges ab.}
\end{tikzpicture}
\hspace{-0.6cm}
\begin{tikzpicture}
	\card
	\cardstrip
	\cardbanner{banner/orange.png}
	\cardicon{icons/coin.png}
	\cardprice{5}
	\cardtitle{\footnotesize{Korsarenschiff}}
	\cardcontent{Das entsorgte \emph{SILBER} oder \emph{GOLD} gibt in seinem Zug trotzdem \coin für den Spieler, der es entsorgt.

	\medskip

	Wenn du \emph{SILBER} und \emph{GOLD} in einem Zug spielst und vom Angriff des \emph{KORSAREN} betroffen bist, wird nur eine Karte davon (die zuerst gespielte) entsorgt. }
\end{tikzpicture}
\hspace{-0.6cm}
\begin{tikzpicture}
	\card
	\cardstrip
	\cardbanner{banner/orange.png}
	\cardicon{icons/coin.png}
	\cardprice{5}
	\cardtitle{Meerhexe}
	\cardcontent{Wenn du diese Karte spielst, ziehst du 2 Karten und jeder Mitspieler nimmt einen \emph{FLUCH}. Zu Beginn deines nächsten Zuges ziehst du 2 Karten und legst dann 2 Karten ab.}
\end{tikzpicture}
\hspace{-0.6cm}
\begin{tikzpicture}
	\card
	\cardstrip
	\cardbanner{banner/orangeblue.png}
	\cardicon{icons/coin.png}
	\cardprice{5}
	\cardtitle{Piratin}
	\cardcontent{Du kannst diese Karte spielen, wenn du eine Geldkarte nimmst, oder wenn ein Mitspieler eine Geldkarte nimmt. Spielst du diese Karte während des Zuges eines Mitspielers, ist dein nächster Zug der Zug, in dem du durch die \emph{PIRATIN} eine Geldkarte nimmst. Die Geldkarte, die du nimmst, kommt aus dem Vorrat und du nimmst sie direkt auf deine Hand.}
\end{tikzpicture}
\hspace{-0.6cm}
\begin{tikzpicture}
	\card
	\cardstrip
	\cardbanner{banner/white.png}
	\cardicon{icons/coin.png}
	\cardprice{5}
	\cardtitle{\footnotesize{Schatzkammer}}
	\cardcontent{Wenn du eine \emph{SCHATZKAMMER} spielst und in deiner Kaufphase \emph{keine} Punktekarte genommen hast, \emph{darfst} du die gespielte \emph{SCHATZKAMMER} am Ende deiner Kaufphase zurück auf den Nachziehstapel legen. 

	\medskip

	Wenn du mehrere \emph{SCHATZKAMMERN} gespielt hast, darfst du auch diese \emph{SCHATZKAMMERN} auf den Nachziehstapel zurücklegen.

	\medskip

	Punktekarten, die du außerhalb der Kaufphase nimmst (wie z.B. durch einen \emph{SCHMUGGLER}) verhindern nicht, dass du \emph{SCHATZKAMMERN} auf deinen Nachziehstapel legst. }
\end{tikzpicture}
\hspace{-0.6cm}
\begin{tikzpicture}
	\card
	\cardstrip
	\cardbanner{banner/orange.png}
	\cardicon{icons/coin.png}
	\cardprice{5}
	\cardtitle{Taktiker}
	\cardcontent{Hast du mindestens eine Handkarte, legst du alle Handkarten ab, ziehst zu Beginn deines nächsten Zuges 5 Karten und erhältst eine zusätzliche Aktion sowie einen zusätzlichen Kauf.
	
	\smallskip

	Hast du aber keine Handkarte, bekommst du im nächsten Zug keinen dieser Boni und der \emph{TAKTIKER} wird bereits in dem Zug abgelegt, in dem er gespielt wurde (denn er hat in diesem Zug seine letzte Anweisung ausgeführt).

	\medskip

	\emph{Grundsätzlich gilt: Nur wenn du mindestens eine Handkarte ablegen kannst, erhältst du den Bonus im nächsten Zug. Wenn du keine Karte ablegen kannst, legst du den TAKTIKER in der Aufräumphase dieses Zuges ab.}

	\medskip

	Wenn du den \emph{TAKTIKER} doppelt spielst (z.B. durch einen \emph{THRONSAAL} aus dem \emph{Basisspiel}), erhältst du den Bonus im nächsten Zug nur einmal, da du beim zweiten Spielen des \emph{TAKTIKERS} keine Handkarte mehr auf der Hand hast und damit die Bedingung nicht erfüllst. }
\end{tikzpicture}
\hspace{-0.6cm}
\begin{tikzpicture}
	\card
	\cardstrip
	\cardbanner{banner/orange.png}
	\cardicon{icons/coin.png}
	\cardprice{5}
	\cardtitle{Werft}
	\cardcontent{Du \emph{musst} 2 Karten ziehen und \emph{darfst} einen weiteren Kauf tätigen. Zu Beginn deines nächsten Zuges (nicht vorher) \emph{musst} du wieder 2 Karten ziehen und erhältst +1 Kauf.}
\end{tikzpicture}
\hspace{-0.6cm}
\begin{tikzpicture}
	\card
	\cardstrip
	\cardbanner{banner/white.png}
	\cardtitle{\footnotesize{Neue Regeln (1/2)}\qquad}
	\cardcontent{\tiny{\begin{Spacing}{1}\vspace{1em}
	Es gelten die Basisspielregeln mit folgenden Änderungen:

	\smallskip

	\emph{Dauerkarten:}\\
	Normalerweise werden alle im Spiel befindlichen Karten in der Aufräumphase aus dem Spielbereich entfernt und auf den Ablagestapel gelegt. Die orangefarbenen \emph{Dauerkarten} beinhalten Anweisungen, die in späteren Zügen umgesetzt werden. Sie werden nicht in der Aufräumphase des Zuges abgelegt, in dem sie gespielt wurden, sondern bleiben bis zur Aufräumphase des Zuges, in dem die letzte Anweisung ausgeführt wird, im Spiel. Wird eine \emph{Dauerkarte} mehrfach gespielt (z.B. durch den \emph{THRONSAAL} aus dem \emph{Basisspiel}), bleibt die verursachende Karte ebenfalls so lange im Spiel, bis die \emph{Dauerkarte} abgelegt wird.

	\smallskip

	Um die Übersicht zu behalten, welche Karte im aktuellen Zug gespielt wurde oder in einem früheren, empfehlen wir folgendes Vorgehen: nach dem Spielen einer Dauerkarte prüfst du, ob ihre Anweisungen für einen zukünftigen Zug wirksam sind. Wenn ja, drehe sie um 90 Grad auf die Seite (quer liegend). Daran siehst du, dass sie in der Aufräumphase des aktuellen Zuges nicht abgelegt wird. Stattdessen verschiebst du sie in der Aufräumphase in den Bereich oberhalb des normalen Spielbereichs. Solange sie nicht abgehandelt ist, bleibt sie dort quer liegen. Sobald alle ihre Anweisungen abgehandelt sind, drehst du sie wieder senkrecht. Daran siehst du dann, dass sie in der Aufräumphase jenes Zuges mit den anderen Karten aus dem normalen Spielbereich abgelegt wird.

	\smallskip

	Bei mehreren im Spiel befindlichen Dauerkarten darfst du die Reihenfolge selbst bestimmen, in der du sie abhandelst.
	\end{Spacing}}}
\end{tikzpicture}
\hspace{-0.6cm}
\begin{tikzpicture}
	\card
	\cardstrip
	\cardbanner{banner/white.png}
	\cardtitle{\footnotesize{Neue Regeln (2/2)}\qquad}
	\cardcontent{\emph{Die kombinierten Königreichkarten:}\\
	Kombinierte Königreichkarten sind Karten, die 2 oder mehr Kartentypen angehören. Ihren Kartentypen entsprechend (\emph{GELD} oder \emph{AKTION}) können sie wie normale Geld- oder Aktionskarten in der entsprechenden Spielphase eingesetzt werden. Jede Anweisung, die sich auf Karten eines Typs beziehen, betrifft auch die entsprechenden kombinierten Königreichkarten.

	\medskip

	Dies gilt auch für die \emph{INSEL}. Sie ist eine Aktions- und Punktekarte und ihre Anzahl im Stapel ist von der Spielerzahl abhängt. Wird die \emph{INSEL} im Spiel verwendet, legt ihr zu der Platzhalterkarte die folgende Anzahl Karten in die Tischmitte:

	\medskip
	\emph{Bei 3 oder mehr Spielern}: 12 Karte\\
	\emph{Bei 2 Spielern}: 8 Karten
	}
\end{tikzpicture}
\hspace{-0.6cm}
\begin{tikzpicture}
	\card
	\cardstrip
	\cardbanner{banner/white.png}
	\cardtitle{\footnotesize{Anweisungen (1/3)}\qquad}
	\cardcontent{\emph{Wähle eins}: Du musst genau eine der Anweisungen auf der Karte auswählen und sie, soweit möglich, ausführen. Es ist erlaubt, eine Anweisung zu wählen, die du nicht ausführen kannst. Die restlichen Anweisungen haben für diesen Spielzug keine Wirkung. Wenn die Karte später im Spiel erneut gespielt wird, darfst du eine andere Wahl treffen.

	\smallskip

	\emph{Ansehen}: Du nimmst die angegebene Karte, siehst sie dir an und legst sie – falls nicht anders auf der Karte angewiesen – dorthin zurück, von wo du sie hast. Du darfst die Karte keinem Mitspieler zeigen.

	\smallskip

	\emph{Im Spiel}: Geld- und Aktionskarten, die du offen in deinem Spielbereich vor dir liegen hast, befinden sich im Spiel, bis sie in der Aufräumphase abgelegt werden. Nicht im Spiel befinden sich zur Seite gelegte und entsorgte Karten, sowie alle Handkarten, Karten im Vorrat und in den Zieh- und Ablagestapeln. Auch Reaktionskarten, die als Reaktion auf eine Angriffskarte aufgedeckt werden, befinden sich nicht im Spiel.

	\smallskip

	\emph{Ablegen}: Karten werden immer von der Hand abgelegt, sofern nicht anders auf der Karte angegeben. Abgelegte Karten kommen offen auf den eigenen Ablagestapel. Legst du mehrere Karten gleichzeitig ab, musst du diese den Mitspielern nicht zeigen. Ggf. musst du aber die Anzahl der abgelegten Karten \enquote{nachweisen}, z.B. beim \emph{KELLER} (aus dem \emph{Basisspiel}). Lediglich die oberste Karte des Ablagestapels muss immer sichtbar sein.}
\end{tikzpicture}
\hspace{-0.6cm}
\begin{tikzpicture}
	\card
	\cardstrip
	\cardbanner{banner/white.png}
	\cardtitle{\footnotesize{Anweisungen (2/3)}\qquad}
	\cardcontent{\emph{Aufdecken}: Du deckst die Karte(n) auf, zeigst sie allen Mitspielern und legst sie dorthin zurück, von wo du sie hast. Eine aufgedeckte Handkarte wird wieder zurück auf die Hand genommen.

	\smallskip

	\emph{Diese Karte}: Enthält eine Karte eine Anweisung, die sich auf \enquote{diese Karte} bezieht, ist normalerweise die Karte gemeint, auf der die Anweisung steht, keine andere Karte, auf die innerhalb der Anweisung Bezug genommen wird. Eine \emph{Ausnahme} sind die \emph{Wege} (aus \emph{Menagerie}), bei denen mit \enquote{diese Karte} die Aktionskarte gemeint ist, anstelle derer die Anweisung der \emph{Wege}-Karte ausgeführt wird. \\
	Beispiel: Der Kartentext der \emph{PIRATIN} besagt: \enquote{Nimmt ein Spieler eine Geldkarte, darfst du diese Karte aus deiner Hand spielen.} Dies bedeutet, dass du, wenn jemand eine Geldkarte nimmt, die \emph{PIRATIN} aus deiner Hand spielen darfst - nicht etwa die Geldkarte.

	\smallskip

	\emph{Jene Karten}: Enthält eine Karte eine Anweisung, die sich auf \enquote{jene Karten} bezieht, sind immer die Karten gemeint, auf die auf einer Karte (\enquote{dieser Karte}) Bezug genommen wird – es ist niemals die gerade genutzte Karte gemeint.}
\end{tikzpicture}
\hspace{-0.6cm}
\begin{tikzpicture}
	\card
	\cardstrip
	\cardbanner{banner/white.png}
	\cardtitle{\footnotesize{Anweisungen (3/3)}\qquad}
	\cardcontent{\emph{Eine gleiche Karte}: Nur Karten, die exakt denselben Namen tragen, gelten als gleiche Karten.

	\smallskip

	\emph{Entsorgen}: Entsorgst du Karten, legst du sie offen auf den Müllstapel bzw. auf die Müllkarte, falls noch keine Karte entsorgt wurde. Entsorgte Karten können nicht wieder gekauft oder genommen werden, es sei denn, eine Karte erlaubt dies. 

	\smallskip

	\emph{Nehmen}: Karten, die durch Kauf oder eine Anweisung auf einer anderen Karte genommen werden, werden von dir physisch an dich genommen und werden dadurch deinem Kartensatz hinzugefügt. Genommene Karten werden (soweit nicht anders auf der Karte angegeben) auf den Ablagestapel gelegt.

	\smallskip

	\emph{Zur Seite legen}: Karten, die durch eine Anweisung zur Seite gelegt werden, befinden sich nicht im Spiel.}
\end{tikzpicture}
\hspace{-0.6cm}
\begin{tikzpicture}
	\card
	\cardstrip
	\cardbanner{banner/white.png}
	\cardtitle{\scriptsize{Empfohlene 10er Sätze\qquad}}
	\cardcontent{\emph{Auf hoher See:}\\
	Ausguck, Bazar, Blockade, Hafen, Insel, Karawane, Korsarenschiff, Lagerhaus, Piratin, Werft

	\smallskip

	\emph{Vergrabene Schätze:}\\
	Affe, Astrolabium, Außenposten, Beutelschneider, Fischerdorf, Leuchtturm, Schatzkarte, Seefahrerin, Seekarte, Taktiker

	\smallskip

	\emph{Griff nach den Sternen} (Seaside (2. Edition) + \textit{Basisspiel (2. Edition)}):\\
	Affe, Ausguck, Beutelschneider, Meerhexe, Schatzkarte, \textit{Dorf}, \textit{Keller}, \textit{Ratsversammlung}, \textit{Töpferei} \textit{Vasall}

	\smallskip

	\emph{Wiederholungen} (Seaside (2. Edition) + \textit{Basisspiel (2. Edition)}):\\
	Außenposten, Karawane, Piratin, Schatzkammer, Seekarte, \textit{Jahrmarkt}, \textit{Miliz}, \textit{Umbau}, \textit{Vorbotin}, \textit{Werkstatt}

	\smallskip

	\emph{Leitstern (Seaside (2. Edition) + \textit{Intrige (2. Edition)}:)}\\
	Affe, Ausguck, Bazar, Gezeitenbecken, Schatzkarte, \textit{Diplomatin}, \textit{Geheimgang}, \textit{Höflinge}, \textit{Trickser}, \textit{Wunschbrunnen}

	\smallskip

	\emph{Küstenwache (Seaside (2. Edition) + \textit{Intrige (2. Edition)}:)}\\
	 Beutelschneider, Insel, Leuchtturm, Seekarte, Werft, \textit{Armenviertel}, \textit{Austausch}, \textit{Handelsposten}, \textit{Handlanger}, \textit{Patrouille}}
\end{tikzpicture}
\hspace{-0.6cm}
\begin{tikzpicture}
	\card
	\cardstrip
	\cardbanner{banner/white.png}
	\cardtitle{\scriptsize{Empfohlene 10er Sätze\qquad}}
	\cardcontent{\emph{Vollgestopft (Seaside (2. Edition) + \textit{Alchemisten/Reiche Ernte}:)}\\
	Hafen, Lagerhaus, Meerhexe, Seefahrerin, Seekarte, \textit{Kräuterkundiger}, \textit{Lehrling}, \textit{Stein der Weisen}, \textit{Vertrauter}, \textit{Weinberg}

	\smallskip

	\emph{Sammler (Seaside (2. Edition) + \textit{Alchemisten/Reiche Ernte}:)}\\
	Blockade, Fischerdorf, Gezeitenbecken, Handelsschiff, Schmuggler, \textit{Bauerndorf}, \textit{Ernte}, \textit{Festplatz}, \textit{Treibjagd}, \textit{Wahrsagerin}
	
	\smallskip

	\emph{Explodierendes Königreich (Seaside (2. Edition) + \textit{Blütezeit}:)}\\
	Ausguck, Außenposten, Fischerdorf, Taktiker, Werft, \textit{Bischof}, \textit{Großer Markt}, \textit{Königshof}, \textit{Stadt}, \textit{Steinbruch}

	\smallskip

	\emph{Nasses Grag (Seaside (2. Edition) + \textit{Dark Ages}:)}\\
	Eingeborenendorf, Korsarenschiff, Müllverwerter, Schatzkammer, Schatzkarte, \textit{Eremit}, \textit{Grabräuber}, \textit{Graf}, \textit{Lumpensammler}, \textit{Ratten}

	\smallskip

	\emph{Bauern (Seaside (2. Edition) + \textit{Dark Ages}:)}\\
	Fischerdorf, Hafen, Insel, Lagerhaus, Leuchtturm, \textit{Armenhaus}, \textit{Landstreicher}, \textit{Mundraub}, \textit{Vogelfreie}, \textit{Waffenkammer}

	\smallskip

	\emph{Insel-Baumeister (Seaside (2. Edition) + \textit{Die Gilden}:)}\\
	Eingeborenendorf, Insel, Müllverwerter, Schatzkammer, Seekarte, \textit{Bäckerin}, \textit{Berater}, \textit{Kaufmannsgilde}, \textit{Platz}, \textit{Steinmetz}}
\end{tikzpicture}
\hspace{-0.6cm}
\begin{tikzpicture}
	\card
	\cardstrip
	\cardbanner{banner/white.png}
	\cardtitle{\scriptsize{Empfohlene 10er Sätze\qquad}}
	\cardcontent{\emph{Fürst Orange (Seaside (2. Edition) + \textit{Abenteuer}:)}\\
	Astrolabium, Fischerdorf, Handelsschiff, Karawane, Seefahrerin, \textit{Amulett}, \textit{Geisterwald}, \textit{Page}, \textit{Sumpfhexe}, \textit{Verlies}, \textit{\underline{Mission}}

	\smallskip

	\emph{Geschenke und Mathoms (Seaside (2. Edition) + \textit{Abenteuer}:)}\\
	Blockade, Hafen, Müllverwerter, Schmuggler, Seefahrerin, \textit{Brückentroll}, \textit{Gefolgsmann}, \textit{Karawanenwächter}, \textit{Kurier}, \textit{Verlorene Stadt}, \textit{\underline{Expedition}}, \textit{\underline{Quest}}
	
	\smallskip

	\emph{In die Enge getrieben (Seaside (2. Edition) + \textit{Empires}:)}\\
	Lager, Müllverwerter, Schmuggler, Taktiker, Werft, \textit{Feldlager/Diebesgut}, \textit{Gladiator/Reichtum}, \textit{Schlösser}, \textit{Wagenrennen}, \textit{Zauberin}, \textit{\underline{Mauer}}, \textit{\underline{Steuer}}

	\smallskip

	\emph{König der Meere (Seaside (2. Edition) + \textit{Empires}:)}\\
	Eingeborenendorf, Hafen, Korsarenschiff, Meerhexe, Piratin, \textit{Archiv}, \textit{Bauernmarkt}, \textit{Lehnsherr}, \textit{Tempel}, \textit{Wilde Jagd}, \textit{\underline{Brunnen}}, \textit{\underline{Erforschen}}

	\smallskip

	\emph{Das neue Schwarz (Seaside (2. Edition) + \textit{Nocturne}:)}\\
	Handelsschiff, Karawane, Korsarenschiff, Seefahrerin, Taktiker, \textit{Geheime Höhle}, \textit{Geisterstadt}, \textit{Plünderer}, \textit{Schuster}, \textit{Sündenpfuhl}

	\smallskip

	\emph{Verbotene Insel (Seaside (2. Edition) + \textit{Nocturne}:)}\\
	Affe, Bazar, Gezeitenbecken, Müllverwerter, Piratin, \textit{Fährtensucher}, \textit{Friedhof}, \textit{Götze}, \textit{Seliges Dorf}, \textit{Tragischer Held}}
\end{tikzpicture}
\hspace{-0.6cm}
\begin{tikzpicture}
	\card
	\cardstrip
	\cardbanner{banner/white.png}
	\cardtitle{\scriptsize{Empfohlene 10er Sätze\qquad}}
	\cardcontent{\emph{Freihandel (Seaside (2. Edition) + \textit{Renaissance}:)}\\
	Außenposten, Blockade, Insel, Schmuggler, Werft, , \textit{Diener}, \textit{Forscherin}, \textit{Frachtschiff}, \textit{Gewürze}, \textit{Schauspieltruppe}, \textit{\underline{Innovation}}

	\smallskip

	\emph{Goldrausch (Seaside (2. Edition) + \textit{Renaissance}:)}\\
	Astrolabium, Eingeborenendorf, Karawane, Müllverwerter, Schatzkarte, \textit{Bildhauerin}, \textit{Erfinder}, \textit{Fahnenträger}, \textit{Freibeuterin}, \textit{Grenzposten}, \textit{\underline{Fruchtwechsel}}, \textit{\underline{Speicher}}
	
	\smallskip

	\emph{Innsmouth (Seaside (2. Edition) + \textit{Menagerie}:)}\\
	Fischerdorf, Gezeitenbecken, Karawane, Leuchtturm, Piratin, \textit{Hexenzirkel}, \textit{Hirtenhund}, \textit{Lastkahn}, \textit{Stallbursche}, \textit{Viehmarkt}, \textit{\underline{Investition}}, \textit{\underline{Weg der Ziege}}

	\smallskip

	\emph{Ruritanien (Seaside (2. Edition) + \textit{Menagerie}:)}\\
	Astrolabium, Außenposten, Gezeitenbecken, Lagerhaus, Taktiker, \textit{Dorfanger}, \textit{Falknerin}, \textit{Kavallerie}, \textit{Kopfgeldjägerin}, \textit{Schlitten}, \textit{\underline{Bündnis}}, \textit{\underline{Weg des Affen}}

	\smallskip

	\emph{Vorausschauendes Denken (Seaside (2. Edition) + \textit{Verbündete}:)}\\
	Beutelschneider, Eingeborenendorf, Lagerhaus, Meerhexe, Taktiker, \textit{Gildemeisterin}, \textit{Irrfahrten}, \textit{Königliche Galeere}, \textit{Wächter}, \textit{Wegelagerer}, \textit{\underline{Höhlenbewohner}}

	\smallskip

	\emph{Schatzsuche (Seaside (2. Edition) + \textit{Verbündete}:)}\\
	Ausguck, Außenposten, Hafen, Schatzkammer, Schatzkarte, \textit{Bastionen}, \textit{Marquis}, \textit{Ortschaft}, \textit{Tausch}, \textit{Unterhändlerin}, \textit{\underline{Marktstädte}}}
\end{tikzpicture}
\hspace{-0.6cm}
\begin{tikzpicture}
	\card
	\cardstrip
	\cardbanner{banner/white.png}
	\cardtitle{Platzhalter\quad}
\end{tikzpicture}
\hspace{0.6cm}
